% Início do preâmbulo
% Onhe é definida as classes, propriedades, etc.
\documentclass[a4paper, 11pt]{article}
% fim do preâmbulo

% início do documento
\begin{document}
	\title{Example 2}
	\author{Yo}
	\date{May 15, 2022}
	\maketitle % printa o título, autor e a data em um formato bom
	
	 
	\section{What's this?} % produz um texto com fonte maior e em negrito
	This is our second document. It contains a title and a section with text.
\end{document}
% fim do documento

% comando em LaTex é dado usando uma barra invetida (\) mais a palavra reservada (ou comando):
% /comando
% Depois dessa delcaração há os argumentos a serem inseridas. Os argumentos entre chaves {...} são mandatórios, e os argumentos em colchetes [...] são opicionais. Assim:
% \comando{argumentos mandatórios} ou
% \comando[argumentos opicionais]{argumentos mandatórios} 